
In the Belousov-Zhabotinsky (BZ) reaction, an acidic environment is necessary to produce \(\text{Br}^-\) ions from sodium bromate, which we achieve by using sulfuric acid. An increase in the concentration of sulfuric acid leads to a higher rate of \(\text{Br}^-\) production. This results in greater availability of \ch{Br^-} ions for the chemical oscillation and self-propelled motion of BZ droplets. The speed of the droplet, the fluctuation in the direction of motion, and the frequency of oscillations inside the droplet increase with the sulphuric acid. The effect of sulphuric acid on the frequency of oscillations was small compared to that of sodium bromate.
%%%%%%%%%%%%%%%figure
\begin{figure}[H]
    \centering
    \includegraphics[width=1\linewidth]{Figurs/Combined_activity_H2SO4.pdf}
    \caption{\textbf{Activity of BZ Droplet with Sulphuric Acid:} Plots a, b, and c are the frequency, speed, and fluctuation in the direction of the BZ droplet with sulphuric acid, respectively. The frequency of oscillations inside the droplet, the speed of motion, and fluctuations in the direction of motion increase with sulphuric acid.}
    \label{Activity_H2SO4}
\end{figure}

We measure the inter-droplet distance of the interacting BZ droplet for different concentrations of sulphuric acid with time. The dotted line presents the interaction time of the BZ droplet when it is 2.5mm or less.
\begin{figure}[H]
    \centering
    \includegraphics[width=1\linewidth]{Figurs/Sulphuric_R12_combined.pdf}
    \caption{\textbf{Inter Droplet Distance with Sulphuric Acid:} Plot (a), (b), and (c) is the distance between the centers of droplets with time for the different sulphuric acid concentrations of 0.5M, 0.7 M, and 0.8M, respectively. The sodium bromate concentration was 0.48M. The dotted line presents the interaction time when the inter-droplet distance was 2.5mm or less for the 1.75mm diameter.}
    \label{H2SO4_R12}
\end{figure}
When the concentration of sulphuric acid increases in the BZ droplet, the interaction time of the BZ droplet decreases. In Figure \ref{fig:dropletDistance}(d), at a higher concentration of sodium bromate, due to chemical wave interaction, the interaction time was not varied much. In this interaction, the effect of the chemical wave was not observed notably in the interaction with sulphuric acid. The interaction decreased linearly.
\begin{figure}[H]
    \centering
    \includegraphics[width=1\linewidth]{Figurs/H2SO4+different_H2SO4_Combined.pdf}
    \caption{\textbf{Interaction Time Interval for Sulphuric Acid:}  Plot (a) is the interaction time for the concentration of \ch{H_2SO_4}. For plot (a), interacting droplets are chemically identical. The interaction time of the BZ droplets decreases with sulphuric acid. Sodium bromate concentration fixed at 0.48M. Plot (b) is the BZ droplet interaction of two BZ droplets with different sulphuric acid concentrations. In one droplet, the sulphuric acid concentration was 0.5M, and in the other droplet, it varied. The interaction time decreases with the concentration of sulphuric acid in the other droplet. Sodium bromate concentration fixed at 0.52M.}
    \label{H2SO4}
\end{figure}
In Figure \ref{H2SO4}, we fixed the concentration of sulphuric acid in one droplet. We varied the concentration of the other droplet to verify the effect of the gradient of sulphuric acid concentration on the interaction of the droplets. The interaction time of the BZ droplets decreased with the increase in the sulphuric acid concentration of the other droplet. From Figure \ ref {H2SO4}(b), we can observe that the interaction time of the BZ droplets with different sulphuric acid concentrations depends more on the droplet with the higher concentration. 