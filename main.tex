\documentclass[singlecolumn]{article}
\usepackage{graphicx} % Required for inserting images
\usepackage[margin=1.5in]{geometry}
\usepackage{chemformula}
\usepackage{subcaption}
\usepackage{times}
\usepackage{graphicx}
\usepackage[font=small]{caption}
\usepackage{float}
\usepackage{todonotes}
\setuptodonotes{inline}
\usepackage{titlesec}   % better modern Times font
\title{Activity Induced Binary Interactions in Self-Propelled Belousov-Zhabotinsky Reaction Droplets}
\author{Vivek Bharat Meshram, and T K Shajahan}
\date{\today}

\begin{document}

\maketitle


\newpage
\section{Introduction}
Self-propelled motion is a common characteristic in many microbial systems. This self-propelled motion in microorganisms is achieved by utilising stored internal energy and harnessing energy from the surrounding environment, converting it into mechanical energy to propel themselves \cite{rosing2005thermodynamics, dixit2023pathway}. There are wide groups of microorganisms, such as E. coli, paramecium, amoeba, and spermatozoa \cite{lauga2009hydrodynamics}, which show the self-propelled motion and use the sophisticated sensing mechanism for the selection of direction of motion, such as chemotaxis, thermotaxis, electrotaxis, magnetotaxis, {\it etc} \cite{kumar2010physics,suarez2006sperm,wagner2024magnetotaxis,hokmabad2022chemotactic,buness2024electrotaxis}. To make a system that can mimic characteristics of microorganisms, researchers designed several active systems, which include Janus particles, camphor boats, micro robots, active droplets, {\it etc.} \cite{gomez2016dynamics,heisler2012swarming, singh2025active}. 
The active droplets often mimic key characteristics of motion observed in unicellular organisms, including the run-and-tumble behaviour of bacteria, the locomotion of amoebas, and other taxis effects observed in bacteria. For instance, the thymol acetate (TA) droplet demonstrates a run-and-tumble motion. On a sodium dodecyl sulphate (SDS) solution buffered at constant pH \cite{fujino2025differentiation}. At the same time, droplets containing an organic solvent and 2-hexyldecanoic acid can be directed by a pH gradient \cite{lagzi2010maze}. The salt concentration drives the translation motion of an alcohol droplet \cite{cejkova2014dynamics} in a sodium deconate solution. Furthermore, the motion of mercury droplets can be influenced by an electric field \cite{hollo2021electric}.

One important class of chemically driven active droplets contains the Belousov-Zhabotinsky (BZ) reaction. The BZ reaction is a complex, nonlinear chemical oscillatory reaction \cite{zaikin1970concentration,zhabotinsky1964periodical}. In this reaction, oscillations are observed as a colour change. The chemical
Oscillations in the BZ reaction can propagate in space as chemical waves, creating circular and spiral patterns \cite{amrutha2022mechanism,amrutha2023theory,sebastian2024effect}. The internal chemical oscillations inside the BZ droplets drive the mechanical motion, converting chemical energy to mechanical energy. Many studies have shown that such chemical waves can affect the direction and magnitude of the self-propelled motion in these droplets \cite{suematsu2016oscillation,back2023electrotaxis}. Several research groups have investigated BZ reaction droplets to understand self-propelled and run-and-tumble motions, the effect of external electric fields, and the interactions among droplets in an oil medium  \cite{back2023electrotaxis,suematsu2021spontaneous,kitahata2011spontaneous,kumar2021fast}. The interaction studies of BZ droplets were conducted in capillary tubes \cite{delgado2011coupled,li2014combined} and in structured arrays \cite{chang2018towards}. Within these interaction experiments, the droplets were not free to move; they were compactly arranged and separated by oil containing surfactants, which prevented them from merging. They have observed that the chemical oscillations can be synchronised
across droplets, and droplets could communicate via chemical signalling.
In this paper, we investigate the interaction between two moving BZ
droplets within a medium of moonolein-squalane oil. Our findings indicate that the sodium bromate concentration can control the activity of the BZ droplet. As the activity determined by sodium bromate concentration increases within a BZ droplet, the interaction time between the two droplets decreases. Furthermore, when two droplets with varying levels of activity come into contact, the interaction time is influenced more significantly by the droplet with the higher activity.
\section{Methods}
The Belousov-Zhabotinsky (BZ) reaction is a chemical process characterised by cyclic oscillations. These oscillations can be seen within a BZ droplet, displaying red and blue colours that correspond to the oxidation states of a metal catalyst. When the metal catalyst is in its oxidized state ($Fe^{4+}$), the BZ reaction exhibits a blue colour, while in its reduced state ($Fe^{3+}$), it appears red. This phenomenon allows us to observe the chemical oscillations within the BZ droplet, as visible in Figure \ref{fig:setup}.
The Belousov-Zhabotinsky reaction is a complex process that consists of three main stages, each involving numerous sub-reactions. In the first stage, there is a continuous consumption of bromide (\ch{Br^-}) ions. Once the concentration of these ions drops below a certain level, the second stage begins. During the second stage, the metal catalyst undergoes oxidation, and the reaction turns blue. The third stage involves a reduction of the metal catalyst, leading to a significant increase in the concentration of \ch{Br^-} ions. As the levels of \ch{Br^-} ions rise sufficiently, the cycle restarts with the first stage. The third stage essentially acts as a reset for the entire reaction \cite{field1974oscillations}. This process can be visualised through the BZ droplet, as shown in Figure \ref{fig:BZ Droplet}.


\begin{figure}[H]
	\centering
	\includegraphics[width=0.5\linewidth]{Figurs/Picture2.png}
	\caption{\textbf{Mechanism of the Belousov Zhabotinsky Reaction  Inside Droplet:} (a) It is the illustration of the self-propelled BZ droplet and the cyclic mechanism of the Belousov-Zhabotinsky chemical reaction inside the droplet. The BZ reaction consists of three stages, each stage showing a different color, corresponding to the oxidation and reduction of the ferroin metal catalyst.}
	\label{fig:BZ Droplet}
\end{figure}




We have used this aqueous droplet of the BZ reaction mixture, immersed in monoolein-squalane oil.
The chemical concentrations of the Belousov-Zhabotinsky stock solution 
are as follows:
0.96M malonic acid (\ch{CH_2(COOH)_2}), 0.97M sodium bromide (NaBr), 
0.5M sulphuric acid (\ch{H_2SO_4}), and
25mM ferroin indicator solution (\ch{[Fe(phe)_3]SO_4}) was used as a
metal catalyst and indicator. We varied the
sodium bromate (\ch{NaBrO_3}) concentration from 0.48M-0.96M. 
We mixed them using a magnetic stirrer with
continuous stirring until the first chemical oscillation was observed. The
time required to get the first oscillation is reduced with the sodium
bromate concentration. After the first oscillation, we placed 2$\mu$L of this mixture
into a 10mM monoolein-squalane oil medium inside a 5cm diameter S-line Petri dish.
The experiments were conducted at room temperature ($24^{\circ}$), 
maintaining consistent environmental conditions
throughout the experiments. 

The movement of the BZ droplets was recorded
using a CMOS camera.
The images were captured at a rate of five frames per second
and analyzed using ImageJ software and the OpenCV library in Python. The spatial coordinates and average color intensity of the BZ
droplets were recorded for each frame. We have examined 
more than 300 droplets. A schematic diagram of the experimental setup is shown
in Figure~\ref{fig:setup}. Each experimental data set consists of an average of 10 interactions.




\begin{figure}[H]
	\centering
	\includegraphics[width=1\linewidth]{Figurs/Illustration_scalebar12.pdf}
	\caption{\textbf{Self-Propelled Motion of Belousov-Zhabotinsky Reaction Droplet:} (a)Schematic diagram of the experimental setup. The BZ droplet was placed inside a Petri dish with a micropipette. The images are captured from the top view with a CMOS camera and transferred to a computer for further analysis. (b) Trajectory of the BZ droplet in Mo-Sq oil medium. (c) The oscillations inside the BZ droplet are observed in the form of blue color. It is the color oscillations inside the BZ droplet in the form of average blue color intensity ( in arbitrary units a.u.) as a function of time. (d-f) is the propagation of the chemical wave inside the BZ droplet with time.}
	\label{fig:setup}
\end{figure}



%\begin{figure}[H]
  %  \centering
    %\includegraphics[width=0.7\linewidth]{Figurs/Illustration_scalebar.pdf}
   % \caption{\textbf{Self-Propelled Motion of BZ Droplet:} (a) It illustrates the
    %	self-propelled BZ droplet motion and propagation of chemical
    %	wave inside the droplet. The direction of the chemical wave
    	%inside the BZ droplet and the self-propelled motion of the BZ
    %	droplet were the same. (b) Trajectory of the BZ droplet in Mo-Sq oil
    %  medium. (c) The oscillations inside the BZ droplet observed in
     % the form of blue color. Therefore the color oscillations inside
     % the droplet are plotted as the average blue color intensity (in
     % arbitrary units a.u.) of the BZ droplet as a function of time.}
  %  \label{fig:enter-label}
%\end{figure}



\section{Results}
% The Belousov-Zhabotinsky reaction is a chemical oscillatory
% reaction. When we put Belousov-Zhabotinsky reaction droplets quite
% separated from each other in a MO-Sq oil medium, all droplets move
% in a different direction. While moving randomly, droplets come
% together and show interesting collective dynamics. When droplets
% move in an sq-mo oil medium, bromination of monoolein occurs at the
% interface of the BZ droplet. The bromination of monoolein prevents
% droplet coalescence during close droplet interaction and stabilizes
% the droplet. In this experiment, we ensured that the number of BZ
% droplets was not too concentrated, and we also disregarded
% interactions involving more than two droplets.

%When an oscillatory Belousov-Zhabotinsky reaction droplet is placed in
%a 10 mM monoolein-squalane oil medium, the BZ droplet shows the self -
%propelled motion on 2D surface of the Petri dish. The motion of the BZ
%droplet was random. The self-propelled motion of the BZ droplet is
%associate with chemical oscillations inside the BZ droplet and
%interfacial tension around the droplet. Inside the BZ droplet droplet,
%Belousov-Zhabotinsky reaction produces the \ch{Br^-} ions, which is
%helpful in the BZ reaction for oscillation and these emitted \ch{Br^-}
%ions also react with the monoolein present in the oil medium and
%reduces the interafacial tension around the BZ droplet \cite{suematsu2016oscillation}. Due to
%reduction in interfacial tension around the BZ droplet, BZ droplet
%moves from lower interfacial tension to higher interfacial tension.
A small droplet containing 2$\mu L$ of the  Belousov-Zhabotinsky reagents is placed in a Petri dish filled with oily medium, and the BZ droplet exhibits self-propelled motion on the 2D surface of the Petri dish. The self-propelled motion of the BZ droplet is associated with chemical oscillations inside the BZ droplet and interfacial tension around the droplet. We measure the dynamics of the BZ droplet as a function of sodium bromate. In Figure \ref{fig:activity}(a-c), we can observe the trajectory of the BZ droplet with varying sodium bromate concentration. As the sodium bromate concentration increases in the BZ droplet, the BZ droplet's trajectory becomes more complex. We track the average intensity of moving BZ droplets as discussed in the method. From this time series, we calculated the average frequency of chemical oscillations inside the BZ droplet. The average frequency of chemical oscillations inside the BZ droplet increases with the sodium bromate concentration.
Additionally, we track how often the BZ droplet's path fluctuated from its straight trajectory during the initial 200 seconds. Although the BZ droplet trajectory was not very linear, we plotted the changes in direction of the BZ droplet at angles greater than {$45^0$}. In Figure \ref{fig:activity}, it is visible that the fluctuation in direction increases with sodium bromate. While moving the BZ droplet, we track the coordinate of the BZ droplet for each frame. From this coordinate, we calculate the instantaneous speed of the BZ droplet, then average these measurements to determine the overall speed of the BZ droplet. We observed that the speed of the BZ droplet increases with the sodium bromate concentration. Figure \ref{fig:activity}(d-f) shows that the frequency of chemical oscillations, fluctuation in direction, and the speed of the BZ droplet increase with sodium bromate concentration. It indicates that the BZ droplet's activity increases as a function of sodium bromate.




\begin{figure}[H]
    \centering
    \includegraphics[width=1\linewidth]{Figurs/Combined_activity.pdf}
    \caption{\textbf{Activity of BZ Droplet with Sodium Bromate:}
    	Figures a, b, and c are the trajectories of the single BZ
    	droplet for the 0.48M, 0.72M, and 0.96M \ch{NaBrO_3}
    	concentration, respectively. We observed that the BZ droplet
    	changed direction more frequently with an increase in
    	\ch{NaBrO_3}. To clarify this, we plot the frequency of
    	oscillations inside the BZ droplet (d) and the number of times
    	the direction changed from its straight path by the BZ
    	droplet(e) as a function of \ch{NaBrO_3}. Plot (d) presents the
    	number of oscillations, which increases with the sodium bromate
    	concentration. Plot(e) presents the number of times the droplet
    	shows deviation from its straight path increases as the
    	concentration of \ch{NaBrO_3} increases. Direction changed
    	plotted for the 200s. Plot (f) presents the average speed of the
    	BZ droplet. The speed of the droplet increases with the addition of sodium
    	bromate. It denotes that the sodium bromate increases the
    	activity of the BZ droplet.}
    \label{fig:activity}
\end{figure}



When several BZ droplets are placed in a Petri dish,
Different BZ droplets move in different directions.
However, we observe that droplets often discover other droplets, come closer to
them, and move around or move parallel for some time, and then move away (see Figure~\ref{fig:dropletDistance}).
The circular motion of one droplet around its pair, or pair droplet motion, is seen when
the inter-droplet distance (the distance between the centres of
droplets) is 2.5 mm or less (see supplementary videos). The bromo-mono-olein layer that forms around each droplet in the squalane-monolein medium can prevent the droplets from merging~\cite{thutupalli2011bilayer}.
We characterised this pairwise interaction by measuring the total time during which the two droplets stay in proximity. The interaction time is measured as the time interval during which the inter-droplet distance is below 2.5 mm (Figure~\ref{fig:dropletDistance}).

% \begin{figure}[H]
%   \centering
%   \includegraphics[width=.8\linewidth]{Figurs/Oscillation.pdf}
%   \caption{\textbf{Number of Oscillations and Turn:} Plots (a)
%   presents the number of oscillations inside the self-propelled BZ
%   droplet in 200s. u p to 0.65M \ch{NaBrO_3} no oscillations
%   observed at initial 200S while after that oscillations increased
%   exponentially. Plot(b) presents the number of times the direction
%   changed by the BZ droplet in 200s. As the concentration of
%   \ch{NaBrO_3} increases, the number of times the BZ droplet changes
%   direction increases. This plot shows the BZ droplet transformation
%   from ballistic to random motion.}
    %\label{fig:enter-label}
    % \end{figure}
    % When we put a single droplet in an SQ-MO oil medium with a lower
    % concentration of \ch{NaBrO_3}, sufficient distance traveled by
    % BZ, the droplet changes direction, while at higher
    % concentrations, it continuously changes direction with the
    % position of the nucleation center of the chemical wave inside
    % the droplet. As the \ch{NaBrO_3} concentration increases, BZ
    % droplet motion shifts from ballistic motion to random motion
    % \cite{suematsu2021spontaneous}.
    % \par Initially, we perform several trials to find out the
    % distance from which droplets are attracted. Initially, we keep
    % the BZ droplet at 6.5mm, 4.5mm, and 3mm from the center of the
    % two BZ droplets. We observed that for 6.5mm and 4.5mm distances,
    % both the BZ droplets moved in different directions, while for
    % 3mm, droplets attracted each other and showed collective motion.
    % \begin{figure}[H]
  %  \centering
    %\includegraphics[width=1\linewidth]{Figurs/Trajectory_Circle.pdf}
    %\caption{\textbf{Close Interaction of BZ Droplet:} Plot presents
    % the trajectory of the close interaction of the BZ droplet for
    % 0.48M \ch{NaBrO_3} concentration with a 2$\mu$l droplet. The
    % diameter of the BZ droplet was 1.75mm. a, b, and c represent
    % before interaction, during interaction, and after interaction,
    % BZ droplet position, respectively.}
    %\label{fig:enter-label}
%\end{figure}

%\begin{figure}[H]
    %\centering
    %\includegraphics[width=1\linewidth]{Figurs/Diff_NaBrO3_trajectories.pdf}
   % \caption{\textbf{Interaction of BZ Droplet :} Plots a, b, and c
    %  are the trajectories of interaction of the BZ droplet for 0.48M,
    %  0.72M, and 0.96M sodium bromate concentration, respectively. The
     % diameter of the BZ droplet was 1.75mm. ${T_0}$, ${T_{1/2}}$,
      %and ${T_0}$ represent the BZ droplets' position before
      %interaction, during interaction, and after interaction,
     % respectively. The arrow presents the direction of self-propelled
      %motion of the BZ droplets.}
    %\label{fig:tango}
%\end{figure}

\begin{figure}[H]
    \centering
    \includegraphics[width=1\linewidth]{Figurs/R12_NaBrO3_trajectories12}
    \caption{\textbf{Interaction of Two BZ Droplets :} (a) The figure is
    	an illustration of the interaction of two BZ droplets. R12 is
    	the distance between the centers of droplets. Arrows present the
    	direction of droplets. (b) It is the interaction of the two BZ droplets and their trajectories. Arrows indicate the direction of BZ droplets. (c) The distance between the center of the interacting BZ droplet and the time. The diameter of the interacting BZ droplets was 1.75mm. The red line denotes the interaction time of the BZ droplet when the distance between the center of the BZ droplet (R12) is 2.5mm or less. (d) Interaction time of the two BZ droplets as a function of sodium bromate concentration. For the A, both the BZ droplet has the same concentration, while for B, both interacting droplets have different sodium bromate concentrations. In one droplet, the BZ droplet sodium concentration was fixed (i.e., \ch{NaBrO_3} = 0.52M ), and varied in other droplets. The interaction time of the BZ droplet decreases with sodium bromide, but at higher concentrations, the change in interaction time was small.}
    \label{fig:dropletDistance}
\end{figure}

%\begin{figure}[H]
   % \centering
    %\includegraphics[width=1\linewidth]{Figurs/R12_NaBrO3_trajectories}
   % \caption{\textbf{Inter Droplet Distance with Sodium Bromate:} Plot
      %a, b, and c present the distance between the center of the
     % interacting droplet with time for the different \ch{NaBrO_3}
    %  concentration. The
%diameter of the interacting droplet was 1.75 mm.The red line denotes the close interaction time
    %  of the BZ droplet when the distance between the center of the
  %    droplet (R12) is less than 2.5mm.}
  %  \label{Inter-Droplet Distance}
%\end{figure}

We measure the interaction time of the two BZ droplets for the
different sodium bromate concentrations. First, we measure the
interaction of chemically identical BZ droplets, in which both
interacting BZ droplets have the same sodium bromate concentration. For
this interaction of the BZ droplet, we gradually vary the sodium
bromate concentration in both BZ droplets. In Figure
\ref{fig:dropletDistance}(d), for A, we observe that the interaction
time of the BZ droplet decreases with the sodium bromate
concentration. However, at higher sodium bromate
concentration, the reduction in interaction time was minimal.
Similarly, we measured the interaction time of the BZ droplets
with different sodium bromate concentrations. We fixed the sodium
bromate concentration in one BZ droplet and varied it in the other BZ
droplet. In Figure \ref{fig:dropletDistance}(d), for B, we can observe
a similar behaviour as we observed for A. It indicates that when two
different activities containing BZ droplets interact, the interaction
time of the BZ droplet depends on the BZ droplet with higher activity
or sodium bromate concentration.
% When the two BZ droplets come together, they show collective
% motion. During this collective motion, bromine ions react with
% monoolein from the oil medium and reduce the interfacial tension in
% that region. Due to a lowering in interfacial surface tension, BZ
% droplets move away after some interval of time.  In acid medium
% \ch{NaBrO_3} produces the \ch{Br^-} ions.  When \ch{NaBrO_3}
% concentration increases, then the concentration of \ch{Br^-} ions
% also increases. Which increases the rate of monoolein
% bromination. Around BZ droplet quickly reduces the interfacial
% surface tension, and the droplet moves away.

%\begin{figure}[H]
 %   \centering
  %  \includegraphics[width=0.8\linewidth]{Figurs/NaBrO3+different_NaBrO3_Combined.pdf}
   % \caption{\textbf{Time Interval of Droplet Close Interaction:} Plot
    %  (a) is the interaction time of the same BZ droplets as a
     % function of sodium bromate. The interaction time of the BZ
      %droplets decreased, but at a higher concentration of sodium
%      bromate, the change was small. Plot (b) is the BZ droplet
 %     interaction time for the two droplets with different
  %    concentrations of sodium bromate. In one droplet, the sodium
   %   bromate concentration was fixed (\ch{NaBrO_3} = 0.52M), and the
    %  other droplet varied. The interaction time of the BZ droplets
     % decreases with the sodium concentration of the other
      %droplets. At the higher concentration of sodium bromate in the
%      other droplet, the change in interaction time was small.}
 %   \label{H2SO4+NaBrO3_Combined_fit}
%\end{figure}

%We were also curious about the BZ droplets interaction with various
%active behaviors. To investigate this, we examined the interaction
%between a BZ droplet with a fixed sodium bromate concentration and
%another droplet with varying concentrations of sodium bromate. We
%observed similar results to those shown in Figure
%\ref{H2SO4+NaBrO3_Combined_fit} (a), where the interaction time
%decreases as the sodium bromate concentration in the second BZ droplet
%increases. The interaction time between the BZ droplets varies
%depending on the droplet has the higher concentration of sodium
%bromate.

% When we sequentially increased the concentration \ch{NaBrO_3}, the
% interfacial surface tension around the BZ droplet decreased fast,
% and the droplet moved away. But at higher concentrations of
% \ch{NaBrO_3}, the frequency of chemical wave propagation increased
% inside the droplet. The direction of chemical wave propagation was
% random. Due to the bromination of monoolein, the BZ droplet moves
% away, but at the same time, because of the chemical wave inside the
% droplet, both droplets try to move away but come close together and
% stay together for a sufficient time. Figure
% \ref{H2SO4+NaBrO3_Combined_fit} (a) shows that in low concentration
% region of \ch{NaBrO_3} with a concentration of \ch{NaBrO_3} (\ch{T_2
% - T_1}) reduces but at higher concentration, because of chemical
% wave (\ch{T_2 - T_1}) was nearly constant. The close droplet
% interaction for the BZ droplet for the different chemical
% compositions of \ch{NaBrO_3} is shown in Figure
% \ref{H2SO4+NaBrO3_Combined_fit}(b). In figure (b), the \ch{NaBrO_3}
% concentration of one droplet was 0.53M, and for the other droplet,
% 0.53M, 0.62M, 0.72, 0.84M, and 0.96M, respectively, to know the
% effect of the gradient of \ch{NaBrO_3} concentration on the close
% interaction of BZ droplet. Plot (b) shows the same behavior as
% observed in Figure \ref{H2SO4+NaBrO3_Combined_fit}(a). In figure
% \ref{H2SO4+NaBrO3_Combined_fit}(b), it is observed that the close
% interaction of BZ droplets depends on the BZ droplet, which has a
% higher concentration of \ch{NaBrO_3} than the lower concentration BZ
% droplet.  Similarly, we studied the interaction of BZ droplets of
% different volumes. We observed two different types of
% interaction. In Figure \ref{1&5_combined}, Plot (a) and (b) show the
% path following interaction. When a big droplet follows a small
% droplet, this interaction lasts for a long time. When two BZ
% droplets show interaction from opposite directions, or a small
% droplet follows a big droplet, then the interaction time of the
% droplets is small compared to the first type, as shown in Figure
% \ref{1&5_combined}. When a small BZ droplet follows the big droplet,
% the big droplet, which has a higher surface area, reduces the large
% interfacial surface tension in the path. Due to lower interfacial
% surface tension, small droplets following big droplet interaction
% will not sustain for a long time, as observed when a big droplet
% follows a small droplet.
%\begin{figure}[H]
   % \centering
    %\includegraphics[width=1\linewidth]{Figurs/1&5_Combined12.pdf}
    %\caption{\textbf{BZ Droplet Interaction for Different Size:} The
    % plot presents the trajectory and the distance between the
    % centers of the droplets when two droplets of different sizes
    % interact. Plots a and b show interaction when a big droplet
    % follows the small droplet. This interaction of droplets lasted
    % for a longer time. Plots (c) and (d) present the trajectory and
    % distance between the center of the droplet when the droplets
    % interact during the parallel motion. This interaction was for a
    % short time compared to (a). In the plot, the interaction was
    % plotted of 1 $\mu$L and 5$\mu$L.}
    %\label{1&5_combined}
%\end{figure}

\begin{figure}[H]
    \centering
    \includegraphics[width=1\linewidth]{Diff_Size_Trajectory_Circle.pdf}
    \caption{\textbf{Different Size BZ Droplet Interaction:} Plots a, b, and c are the interaction trajectories of the two different-sized droplets when both droplets interact through parallel motion, opposite direction, and path following motion. Interaction type c had a higher time of interaction than interaction types a and b. }
    \label{Different_size}
\end{figure}

We also investigated how droplets of different sizes interact with one
another. We observed three distinct types of interaction based on how
the droplets approach each other. The droplets can either arrive from
opposite directions, move in parallel, or have a larger droplet trail
behind a smaller one. The larger BZ droplet has a higher speed than the smaller BZ droplet. When
a small droplet approaches from the opposite direction, or when both
droplets move parallel while maintaining a minimum separation, the
interaction time is shorter than when a larger droplet follows a
smaller one. In the case where the large droplet trails the small one,
its speed is suppressed by the smaller droplet, leading to a sustained
interaction for a long time (Figure \ref{Different_size}).
\begin{figure}[H]
    \centering
    \includegraphics[width=0.8\linewidth]{Figurs/Barplot.pdf}
    \caption{\textbf{Interaction Time of Different-sized Droplets
    		Interactions:} The plots present the interaction time of the interactions among the
    	different-sized droplets. The interaction between a and b is the interaction when droplets arrive from opposite directions, or
    	move parallel with minimum distance. The other part contained
    	the interactions, which had an interaction time of 50 seconds or
    	more. In this interaction region, a small droplet was always
    	followed by a big droplet, as observed in Figure
    	\ref{Different_size}(c).}
    \label{fig:Big-Smal}
\end{figure}


When two droplets with different sizes approach from opposite
directions and move parallel to each other, their interaction time is
typically less than 50 seconds, as we can observe in Figure \ref{fig:Big-Smal}. Conversely, the interaction time exceeds 50 seconds when a
larger droplet trails a smaller one. When the smaller droplet follows
the larger one, the interaction time is negligible. The larger
the droplet is moving at a higher speed. When a smaller droplet follows the larger droplet, the larger droplet easily moves away from interaction, and interaction is not sustained. We have recorded a total of 53 interactions for the different-sized droplet interactions. We observed 22 percent parallel droplet motion. In 45 percent of the interactions, droplets arrived from opposite directions, while in 32 percent of the interactions, a pattern of falling droplets followed the interaction. Additionally, when two droplets of
equal size interact, the time of interaction remains consistent
regardless of their approach sides.
%\begin{figure}[H]
%  \centering
%  \includegraphics[width=0.6\linewidth]{Figurs/Comparison.pdf}
%  \caption{\textbf{Number of Interactions of Both Types:} The plot
%  presents the number of interactions of long-time interaction and
%  short-time interaction out of 100\%. Here, 100\% is the 30 number
%  of interactions. The short-time interaction dominates in the number
%  of times interaction, while long-time interaction dominates in
%  interaction time.}
    %\label{Interaction Bar Plot}
%\end{figure}
%We analyzed the close droplet interaction for BZ droplet for the
% different chemical compositions of \ch{NaBrO_3}as shown in figure
% \ref{different_chemicals}. In figure (a), the \ch{NaBrO_3}
% concentration of one droplet was 0.53M, and for the other droplet,
% 0.53M, 0.62M, 0.72, 0.84M, and 0.96M, respectively. Plot (a) shows
% the same behavior as observed in figure
% \ref{H2SO4+NaBrO3_Combined_fit}. In figure
% \ref{different_chemicals}, it is observed that the close interaction
% of BZ droplets depends on the BZ droplet, which has a higher
% concentration of \ch{NaBrO_3} than the lower concentration BZ
% droplet.

%\begin{figure}[H]
%  \centering
%  \includegraphics[width=1\linewidth]{Figurs/2H2SO4+2NaBrO3_Combined.png}
%  \caption{\textbf{BZ Droplet Interaction for Different Chemical
%  Concentration
%  Containing %Droplet:} Plot (a) presents the 2 $\mu$l BZ droplet interaction time interval for the 0.53M and varying concentration of \ch{NaBrO_3}. Plot (b) shows the 2$\mu$l BZ droplet interaction for the BZ droplet, which is 0.5M and different varying concentrations of \ch{H_2SO_4}.}
    %\label{different_chemicals}
%\end{figure}

\section{Discussion}

It is well known that the chemical oscillations inside the BZ droplet
and the interfacial tension around the BZ droplet causes the
self-propelled motion of the BZ droplet. The change in interfacial
tension around the BZ droplet is influenced significantly by the
surfactant monoolein. The \ch{Br^-} ions produced in the BZ droplet
react with monoolein from the oil medium, and this bromination of
monoolein reduced the interfacial tension around the BZ droplet. Sodium bromate is a source of \ch{Br^-} ions in the BZ reaction. With higher concentrations of sodium bromate, the BZ reaction produced a
large number of \ch{Br^-} ions. When the sodium bromate concentration
increases in the BZ droplet, the rate of bromination of monoolein
increases around the surface of the BZ droplet, and as a result, the speed of the BZ droplet increases \cite{suematsu2021spontaneous}.
In the interaction of two BZ droplets moving in different directions, they come together and interact. How they detect each other from a long distance is unclear. When the two BZ come close, then they interact; their interaction sustains
until the surrounding interfacial tension is reduced
sufficiently. After the sufficient reduction of interfacial tension around both the BZ droplets, they move away from each other. When the sodium bromate concentration increased in the
BZ droplets, the rate of reduction in interfacial tension around BZ
droplets increased due to the bromination of surfactant monoolein from the
oil. As a result, the interaction time of the BZ droplet decreased
with the sodium bromate concentration.

\section{Conclusion}
The self–propelled motion of the BZ droplet is associated with
chemical oscillations inside the droplet and the interfacial tension
around the BZ droplet. The sodium bromate concentration inside the BZ
droplet plays a vital role in the self-propelled motion of the BZ
droplet. In the self-propelled motion of the BZ droplet, fluctuations
in the direction of motion, frequency of oscillations, and speed of
droplet increase as a function of sodium bromate. It confirmed that
the activity of the BZ droplet increases with the sodium bromate
concentration. In this experimental work, we explored the interaction
of BZ droplets as a function of sodium bromate. Two BZ droplets,
during self-propelled motion, move in a random direction, come
together, and show the collective interaction behavior. This BZ droplet
shows the collective interaction and motion, with the inter-droplet
distance of around 2.5 mm or less. The interaction time of the BZ
droplets decreases with the sodium bromate concentration. To explore
the interaction of the BZ droplet with different activities, we fixed
the sodium bromate concentration of one droplet and varied it in the
other BZ droplet; the interaction time decreases as the sodium
concentration increases in the other droplet. The interaction time of
the droplets depends on the droplet with the higher sodium bromate concentration. When
two BZ droplets of different sizes interact, three types of
interactions are observed. Two BZ droplets arrived from opposite
directions, moved in parallel, and the big droplet followed the small
droplet. The first two have the interaction time less than 50
seconds. When a big droplet follows a small droplet, the interaction
time is always higher. The interaction time of the BZ droplet depends
on the droplet with the highest activity. The big droplet has a higher
speed, but when they follow a small droplet, their activity gets
suppressed, and the interaction lasts for a long time.
\par Similarly, BZ droplet activity can be controlled by the sulphuric
acid in the BZ droplet. In an acidic medium, sodium bromate produces
\ch{Br^-} ions. These \ch{Br^-} ions are helpful for the
self-propelled motion and oscillations of the BZ droplet. We observed
nearly the same result with sulphuric acid as we observed with sodium bromate. The interaction time of
the BZ droplet decreases with the sulphuric acid. When two droplets
with different sulphuric acid concentrations interact, the interaction
time depends more on the sulphuric acid concentration of the droplet.

\par In this work, we explored the interaction of BZ droplets when
both droplets are sufficiently close. To know what forces attract the
droplet from a long distance, it is still not clear. We hope
our findings might help researchers to simulate the interaction of an
active system by considering the internal energy of the active system.
% The self–propelled motion of the BZ droplet is associated with
% chemical composition inside the droplet, and the interfacial tension
% around the BZ droplet. Inside the droplet, the BZ reaction produces
% Br- ions, which react with monoolein, which is present in an oil
% medium. Because of the bromination of monoolein, interfacial surface
% tension reduces, and the droplet moves from a lower interfacial
% surface tension to a higher interfacial surface tension. While
% moving randomly, two droplets come close and show collective
% behavior. In this work, we explore the close interaction of BZ
% droplets.  Due to the bromination of monoolein from both droplets,
% when the interfacial tension reduces sufficiently, the droplets move
% away from each other. The concentration of the sodium bromate
% increases the concentration of Br- ions, which leads to an increased
% rate of bromination of monoolein and a decrease in close interaction
% time (\ch{T12}). At higher concentrations of \ch{NaBrO_3}, the
% frequency of chemical oscillations increases. The direction of
% propagation of the chemical wave was random. When two droplets with
% higher concentrations were interacting because of interfacial
% surface tension, droplets tried to move away, but at the same time,
% because of chemical wave orientation, they came together and showed
% collective motion, even maintaining some minimum distance. With the
% sodium bromate concentration, close interaction time decreases, but
% at higher concentrations, due to chemical waves, it becomes nearly
% constant. Similarly, in a higher acidic medium, large Br- ions are
% produced in the BZ reaction, which increases the bromination of
% monoolein. The close interaction time of the BZ reaction is nearly
% reduced with \ch{H_2SO_4}.In the ballistic motion of the BZ droplet,
% the close interaction time interval was maximum, while in the random
% motion, the close interaction was lower compared to the ballistic
% motion. In different-sized droplets, when a big droplet is a small
% droplet, then it is for a long time, while other interactions last
% for a shorter time. Based on this study, we can infer that the
% internal chemical reactions significantly influence the collective
% movement of droplets in the presence of a chemical gradient. We
% anticipate that the close interactions within the active BZ droplet
% system will serve as a valuable model for investigating the
% interactions of biomimetic soft swimmers.

%\newpage
\section{Appendix}

In the Belousov-Zhabotinsky (BZ) reaction, an acidic environment is necessary to produce \(\text{Br}^-\) ions from sodium bromate, which we achieve by using sulfuric acid. An increase in the concentration of sulfuric acid leads to a higher rate of \(\text{Br}^-\) production. This results in greater availability of \ch{Br^-} ions for the chemical oscillation and self-propelled motion of BZ droplets. The speed of the droplet, the fluctuation in the direction of motion, and the frequency of oscillations inside the droplet increase with the sulphuric acid. The effect of sulphuric acid on the frequency of oscillations was small compared to that of sodium bromate.
%%%%%%%%%%%%%%%figure
\begin{figure}[H]
    \centering
    \includegraphics[width=1\linewidth]{Figurs/Combined_activity_H2SO4.pdf}
    \caption{\textbf{Activity of BZ Droplet with Sulphuric Acid:} Plots a, b, and c are the frequency, speed, and fluctuation in the direction of the BZ droplet with sulphuric acid, respectively. The frequency of oscillations inside the droplet, the speed of motion, and fluctuations in the direction of motion increase with sulphuric acid.}
    \label{Activity_H2SO4}
\end{figure}

We measure the inter-droplet distance of the interacting BZ droplet for different concentrations of sulphuric acid with time. The dotted line presents the interaction time of the BZ droplet when it is 2.5mm or less.
\begin{figure}[H]
    \centering
    \includegraphics[width=1\linewidth]{Figurs/Sulphuric_R12_combined.pdf}
    \caption{\textbf{Inter Droplet Distance with Sulphuric Acid:} Plot (a), (b), and (c) is the distance between the centers of droplets with time for the different sulphuric acid concentrations of 0.5M, 0.7 M, and 0.8M, respectively. The sodium bromate concentration was 0.48M. The dotted line presents the interaction time when the inter-droplet distance was 2.5mm or less for the 1.75mm diameter.}
    \label{H2SO4_R12}
\end{figure}
When the concentration of sulphuric acid increases in the BZ droplet, the interaction time of the BZ droplet decreases. In Figure \ref{fig:dropletDistance}(d), at a higher concentration of sodium bromate, due to chemical wave interaction, the interaction time was not varied much. In this interaction, the effect of the chemical wave was not observed notably in the interaction with sulphuric acid. The interaction decreased linearly.
\begin{figure}[H]
    \centering
    \includegraphics[width=1\linewidth]{Figurs/H2SO4+different_H2SO4_Combined.pdf}
    \caption{\textbf{Interaction Time Interval for Sulphuric Acid:}  Plot (a) is the interaction time for the concentration of \ch{H_2SO_4}. For plot (a), interacting droplets are chemically identical. The interaction time of the BZ droplets decreases with sulphuric acid. Sodium bromate concentration fixed at 0.48M. Plot (b) is the BZ droplet interaction of two BZ droplets with different sulphuric acid concentrations. In one droplet, the sulphuric acid concentration was 0.5M, and in the other droplet, it varied. The interaction time decreases with the concentration of sulphuric acid in the other droplet. Sodium bromate concentration fixed at 0.52M.}
    \label{H2SO4}
\end{figure}
In Figure \ref{H2SO4}, we fixed the concentration of sulphuric acid in one droplet. We varied the concentration of the other droplet to verify the effect of the gradient of sulphuric acid concentration on the interaction of the droplets. The interaction time of the BZ droplets decreased with the increase in the sulphuric acid concentration of the other droplet. From Figure \ ref {H2SO4}(b), we can observe that the interaction time of the BZ droplets with different sulphuric acid concentrations depends more on the droplet with the higher concentration. 
%%%%%%%%% bibliography

\bibliographystyle{unsrt}
\bibliography{Reference}

\end{document}
