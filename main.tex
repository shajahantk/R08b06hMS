\documentclass{article}
\usepackage{graphicx} % Required for inserting images
\usepackage[margin=1.5in]{geometry}
\usepackage{chemformula}
\usepackage{subcaption}
\usepackage{times}
\usepackage{graphicx}
\usepackage[font=small]{caption}
\usepackage{float}
\usepackage{todonotes}
\usepackage{titlesec}   % better modern Times font

\title{Activity Induced Binary Interactions in Self-Propelled Belousov-Zhabotinsky Reaction Droplets}
%\author{Vivek Bharat Meshram, and T K Shajahan}
\date{6 Oct 2025}

\begin{document}

\maketitle

\section{Introduction}

Many living and non-living systems, as simple as single cells and
small droplets of liquid, can harness energy from the surroundings and
use it to propel themselves~\cite{rosing2005thermodynamics,
  dixit2023pathway}. The self-propelled motion of droplets often
mimics key characteristics of motion observed in unicellular
organisms, including the run and tumble behavior of bacteria, the
locomotion of amoebas, electrotaxis, magnetotaxis, and chemotaxis
effect observed in bacteria. For instance, a chemically oscillating
droplet exhibiting higher excitability demonstrates a run and tumble
motion \cite{suematsu2021spontaneous}, while droplets containing an
organic solvent and 2-hexyldecanoic acid can be directed by a pH
gradient \cite{lagzi2010maze} also the salt concentration drives
translation motion of alcohol droplet\cite{cejkova2014dynamics} in an
sodium deconate solution. Furthermore, the motion of mercury droplets
can be influenced by an electric field \cite{hollo2021electric}.
  
One important class of chemically driven active droplets contains the
the Belousov-Zhabotinsky (BZ) reaction.  The BZ reaction is a
nonlinear chemical oscillatory reaction
\cite{zaikin1970concentration,zhabotinsky1964periodical}; in this
reaction, oscillations can be observed in the form of a color
change. The internal chemical oscillations inside the BZ droplets
drives the mechanical motion, converting chemical energy to mechanical
energy.  Chemical oscillations in the BZ reaction can propagate in
space as chemical waves, creating circular and spiral
patterns~\cite{amrutha2022mechanism, amrutha2023theory,
  sebastian2024effect}.  Many studies have shown that such chemical
waves can affect the direction and magnitude of the self-propelled
motion in these
droplets~\cite{back2023electrotaxis,suematsu2016oscillation}.

Recently, several research groups have investigated BZ reaction
droplets to understand self-propelled and run-and-tumble motions
\cite{suematsu2021spontaneous}, effect of external electric fields
\cite{back2023electrotaxis} and the interactions among droplets. This
behaviors commonly found in micro biological systems.  Oliver et
al.\cite{steinbock1998radius} were the first to prepare BZ droplets to
study the radius-dependent chemical oscillations within BZ
droplets. Since then, numerous studies have focused on the
self-propelled motion of BZ droplets in an oil medium
\cite{kitahata2011spontaneous, suematsu2016oscillation, kumar2021fast,
  suematsu2021spontaneous} and their interactions. These interaction
studies were conducted in capillary tubes \cite{delgado2011coupled,
  li2014combined} and in structured arrays
\cite{chang2018towards}. Within these interaction experiments, the
droplets were not free to move; they were compactly arranged and
separated by oil containing surfactants. In their studies, Delgado et
al. and Li et al. observed synchronization of chemical oscillations,
noting that while the oscillations start off out of phase, they
eventually synchronize. This synchronization is influenced by the
spacing of the droplets, the presence of malonic acid, and the
illumination conditions in the photosensitive BZ reaction. Similarly,
Chang et al. employed an array with a distinct structure, where the
chemical oscillation in one droplet initiated oscillations in adjacent
droplets, leading to observable chemical communication.


In this paper, we investigate the interaction between two moving BZ
droplets within a medium of moonolein-squalane oil. Our findings
indicate that the interaction can be control based on the chemical
activity of each droplet. We observed that as the activity of the
droplets increases, the interaction time between the BZ droplets
decreases. Furthermore, when two droplets with varying levels of
activity come into contact, the interaction time is influenced more
significantly by the droplet with the higher activity.

\section{Methods}

We conducted experiments at $24^0$ celsius temperature, maintaining
consistent environmental conditions and the experimental setup
throughout the experiments. We prepared the Belousov-Zhabotinsky
reaction with chemical concentrations of the components were 0.96M
malonic acid (\ch{CH_2(COOH)_2}, and 0.97M sodium bromide (NaBr). A
25mM ferroin indicator solution ([Fe$(phe)_3$]S$O_4$) was used as a
metal catalyst for the BZ reaction. In the experiment, we varried the
sodium bromate (\ch{NaBrO_3}) concentration from 0.48M-0.96M in a 0.5M
sulphuric acid. The reaction was performed on a magnetic stirrer with
contineous stirring until first chemical oscillation observed. The
time required to get the first oscillation is reduced with the sodium
bromate concentration. After the first oscillation, we make 2$\mu$L BZ
dropletwith the help of micropipett to pipetted out on a 10mM
monoolein-squalane oil medium in a 5cm diameter S-line petridish, to
investigate the self-propelled motion and interaction of the BZ
droplet.

The self-propelled motion of the BZ droplets were recorded
using a CMOS camera, which was operated through LabVIEW software. We
adjusted the frame rate and pixel size of the images with the LabVIEW
program. The images were captured at a rate of five frames per second
and analyzed using ImageJ software and the OpenCV library in Python
program. The spatial coordintes and average color intensity of the BZ
droplets recorded for the each frame. This experimental work involved
more than 300 droplets.

\begin{figure}[H]
	\centering
	\includegraphics[width=0.5\linewidth]{Figurs/21.png}
	\caption{\textbf{Self-Propelled BZ Droplet:} Figure illustrates the
		experimental setup for the self-propelled motion of the BZ
		droplet. The BZ droplet was placed inside a Petri dish with a
		micropipette. The images are captured from the top view with a
		CMOS camera and transferred to a computer for further
		analysis. }
	\label{Chemical wave direction}
\end{figure}

\begin{figure}[H]
    \centering
    \includegraphics[width=1\linewidth]{Figurs/Illustration_scalebar1.pdf}
    \caption{\textbf{Self-Propelled Motion of BZ Droplet:} (a) It illustrates the
    	self-propelled BZ droplet motion and propagation of chemical
    	wave inside the droplet. The direction of the chemical wave
    	inside the BZ droplet and the self-propelled motion of the BZ
    	droplet were the same. (b) Trajectory of the BZ droplet in Mo-Sq oil
      medium. (c) The oscillations inside the BZ droplet observed in
      the form of blue color. Therefore the color oscillations inside
      the droplet are plotted as the average blue color intensity (in
      arbitrary units a.u.) of the BZ droplet as a function of time.}
    \label{fig:enter-label}
\end{figure}



\section{Results}
% The Belousov-Zhabotinsky reaction is a chemical oscillatory
% reaction. When we put Belousov-Zhabotinsky reaction droplets quite
% separated from each other in a MO-Sq oil medium, all droplets move
% in a different direction. While moving randomly, droplets come
% together and show interesting collective dynamics. When droplets
% move in an sq-mo oil medium, bromination of monoolein occurs at the
% interface of the BZ droplet. The bromination of monoolein prevents
% droplet coalescence during close droplet interaction and stabilizes
% the droplet. In this experiment, we ensured that the number of BZ
% droplets was not too concentrated, and we also disregarded
% interactions involving more than two droplets.

When an oscillatory Belousov-Zhabotinsky reaction droplet is placed in
a 10 mM monoolein-squalane oil medium, the BZ droplet shows the self -
propelled motion on 2D surface of the petridish. The motion of the BZ
droplet was random. The self-propelled motion of the BZ droplet is
associate with chemical oscillations inside the BZ droplet and
interfacial tension around the droplet. Inside the BZ droplet droplet,
Belousov-Zhabotinsky reaction produces the \ch{Br^-} ions, which is
helpful in the BZ reaction for oscillation and these emitted \ch{Br^-}
ions also react with the monoolein present in the oil medium and
reduces the interafacial tension around the BZ droplet \cite{suematsu2016oscillation}. Due to
reduction in interfacial tension around the BZ droplet, BZ droplet
moves from lower interfacial tension to higher interfacial tension.

\begin{figure}[H]
    \centering
    \includegraphics[width=1\linewidth]{Figurs/Combined_activity.pdf}
    \caption{\textbf{Activity of BZ Droplet with Sodium Bromate:}
      Figures a, b, and c are the trajectories of the single BZ
      droplet for the 0.48M, 0.72M, and 0.96M \ch{NaBrO_3}
      concentration, respectively. We observed that the BZ droplet
      changed the direction more frequently with an increase in
      \ch{NabrO_3}. To clarify this, we plot the frequency of
      oscillations inside the BZ droplet (d) and the number of times
      the direction changed from its straight path by the BZ
      droplet(e) as a function of \ch{NabrO_3}. Plot (d) presents the
      number of oscillations, which increases with the sodium bromate
      concentration. Plot(e) presents the number of times the droplet
      shows deviation from its straight path increases as the
      concentration of \ch{NaBrO_3} increases. Direction changed
      plotted for the 200s. Plot (f) presents the average speed of the
      BZ droplet. The speed of the droplet increases with sodium
      bromate. It denotes that the sodium bromate increases the
      activity of the BZ droplet.}
    \label{Activity}
\end{figure}

We investigate the self-propelled motion of BZ droplets with sodium
bromate concentrations varying from 0.48M to 0.96M, while maintaining
constant concentrations of other components in the BZ reaction. Figure
\ref{Activity}(a) illustrates the trajectory of the 2µL BZ droplet at
these different sodium bromate concentrations. As the concentration of
sodium bromate increases, the BZ droplet’s trajectory becomes
increasingly complex. In the plot, we have represented the trajectory
by mapping the coordinates of each frame. With higher sodium bromate
concentrations in the BZ droplet, the reaction rate increases,
resulting in a rise in the frequency of chemical oscillations within
the droplet. We determine the oscillation frequency by counting the
number of oscillations recorded during the 16-minute
experiment. Figure \ref{Activity}(d) shows that the oscillation
frequency is increased with sodium bromate
concentration. Additionally, we tracked how often the droplet's path
fluctuated from its straight trajectory during the initial 200 seconds
of motion. This directional fluctuation also increases with higher
sodium bromate concentrations. We measured the instantaneous speed of
the BZ droplet every second, then averaged these measurements to
determine the overall speed of the BZ droplet. We observed that the
speed of the BZ droplet increases with the sodium bromate
concentration. Figure \ref{Activity} illustrates that the oscillation
frequency, directional fluctuations, and droplet speed all increase
with sodium bromate concentration. This confirms that BZ droplets
exhibit higher activity at higher sodium bromate concentrations.

\par When BZ droplets are placed in a petri dish with a lower density
of BZ droplets, the BZ droplets move in a random direction. During the
self-propelled motion, two droplets come together and show collective
interaction. This interesting collective interaction was observed when
the inter-droplet distance (the distance between the centers of
droplets) was 2.5mm or less. When the inter-droplet distance was 2.5mm
or less, the droplets experienced the influence of each other. During
interaction, droplets touched the boundaries and moved around each
other; additionally, droplets moved together while maintaining an
inter-droplet distance of 2.5 or less (ref. video). While touching
each other's boundaries, coalescence behavior was not observed due to
the presence of the bromomonoolein layer \cite{thutupalli2011bilayer}
on the surface of the BZ droplet. In this experimental work, we
investigated the interaction of two BZ droplets as a function of the
activity of the BZ droplet, in other words, the dependence on sodium
bromate concentrations.
% \begin{figure}[H]
%   \centering
%   \includegraphics[width=.8\linewidth]{Figurs/Oscillation.pdf}
%   \caption{\textbf{Number of Oscillations and Turn:} Plots (a)
%   presents the number of oscillations inside the self-propelled BZ
%   droplet in 200s. u p to 0.65M \ch{NaBrO_3} no oscillations
%   observed at initial 200S while after that oscillations increased
%   exponentially. Plot(b) presents the number of times the direction
%   changed by the BZ droplet in 200s. As the concentration of
%   \ch{NaBrO_3} increases, the number of times the BZ droplet changes
%   direction increases. This plot shows the BZ droplet transformation
%   from ballistic to random motion.}
    %\label{fig:enter-label}
    % \end{figure}
    % When we put a single droplet in an SQ-MO oil medium with a lower
    % concentration of \ch{NaBrO_3}, sufficient distance traveled by
    % BZ, the droplet changes direction, while at higher
    % concentrations, it continuously changes direction with the
    % position of the nucleation center of the chemical wave inside
    % the droplet. As the \ch{NaBrO_3} concentration increases, BZ
    % droplet motion shifts from ballistic motion to random motion
    % \cite{suematsu2021spontaneous}.
    % \par Initially, we perform several trials to find out the
    % distance from which droplets are attracted. Initially, we keep
    % the BZ droplet at 6.5mm, 4.5mm, and 3mm from the center of the
    % two BZ droplets. We observed that for 6.5mm and 4.5mm distances,
    % both the BZ droplets moved in different directions, while for
    % 3mm, droplets attracted each other and showed collective motion.
    % \begin{figure}[H]
  %  \centering
    %\includegraphics[width=1\linewidth]{Figurs/Trajectory_Circle.pdf}
    %\caption{\textbf{Close Interaction of BZ Droplet:} Plot presents
    % the trajectory of the close interaction of the BZ droplet for
    % 0.48M \ch{NaBrO_3} concentration with a 2$\mu$l droplet. The
    % diameter of the BZ droplet was 1.75mm. a, b, and c represent
    % before interaction, during interaction, and after interaction,
    % BZ droplet position, respectively.}
    %\label{fig:enter-label}
%\end{figure}

\begin{figure}[H]
    \centering
    \includegraphics[width=1\linewidth]{Figurs/Diff_NaBrO3_trajectories.pdf}
    \caption{\textbf{Interaction of BZ Droplet :} Plots a, b, and c
      are the trajectories of interaction of the BZ droplet for 0.48M,
      0.72M, and 0.96M sodium bromate concentration, respectively. The
      diameter of the BZ droplet was 1.75mm. \ch{T_0}, \ch{T_{1/2}},
      and \ch{T_0} represent the BZ droplets' position before
      interaction, during interaction, and after interaction,
      respectively. The arrow presents the direction of self-propelled
      motion of the BZ droplets.}
    \label{BZ Droplet Interaction Trajectory}
\end{figure}
In Figure \ref{BZ Droplet Interaction Trajectory}, we can observe the
collective interaction of two BZ droplets with different sodium
bromate concentrations at different times and positions of the
droplets. During the interaction, the BZ droplets move around each
other and also move parallel to each other by maintaining a minimum
distance. After a sufficient amount of time, the droplets move away
from each other. After some time, while moving randomly, both droplets
can interact again, but it is not always true.

\par We tracked the inter-droplet distance (Distance between the
center of the BZ droplet) of the two interacting BZ droplets with
time, as we can see in figure \ref{Inter-Droplet Distance}. The
diameter of the interacting droplet was 1.75 mm. The red dotted line
presents the interaction time of the BZ droplet when the inter-droplet
distance (R12) was 2.5mm or less. We measure the inter-droplet
distance of the BZ droplet for the different sodium bromate
concentrations.
\begin{figure}[H]
    \centering
    \includegraphics[width=0.5\linewidth]{Figurs/R12.png}
    \caption{\textbf{Interaction of Two BZ Droplets :} The figure is
      an illustration of the interaction of two Bz droplets. R12 is
      the distance between the centers of droplets. Arrows present the
      direction of droplets.}
    \label{Interaction Illustration}
\end{figure}

\begin{figure}[H]
    \centering
    \includegraphics[width=1\linewidth]{Figurs/NaBrO3_R12_combined.pdf}
    \caption{\textbf{Inter Droplet Distance with Sodium Bromate:} Plot
      a, b, and c present the distance between the center of the
      interacting droplet with time for the different \ch{NaBrO_3}
      concentration. The red line denotes the close interaction time
      of the BZ droplet when the distance between the center of the
      droplet (R12) is less than 2.5mm.}
    \label{Inter-Droplet Distance}
\end{figure}

We measure the interaction time of the two BZ droplets for the
different sodium bromate concentrations. First, we measure the
interaction of chemically identical BZ droplets, in which both
interacting BZ droplets have the same sodium bomate concentration. For
this interaction of the BZ droplet, we gradually vary the sodium
bromate concentration in both BZ droplets. In figure
\ref{H2SO4+NaBrO3_Combined_fit}(a), we observe that the interaction
time of the BZ droplet decreases with the sodium bromate
concentration. At the higher concentration of sodium bromate, the
effect of chemical wave propagation inside the droplet increased. The
chemical wave propagation inside the BZ droplet was very frequent, and
the orientation was random. When two interacting droplets, due to a
decrease in interfacial tension, move away from each other
simultaneously as a result of the propagation of a chemical wave
within the droplets, they subsequently come together and exhibit
collective behavior. We can observe that in Figure
\ref{H2SO4+NaBrO3_Combined_fit} (a), at higher sodium bromate
concentration, the reduction in interaction time was very
small. Similarly, we measured the interaction time of the BZ droplets
with different sodium bromate concentrations. We fixed the sodium
bromate concentration in one BZ droplet and varied it in the other BZ
droplet. In figure \ref{H2SO4+NaBrO3_Combined_fit}(b), we can observe
a similar behavior as we observed in figure
\ref{H2SO4+NaBrO3_Combined_fit}(a). It indicates that when two
different activities containing BZ droplets interact, the interaction
time of the BZ droplet depends on the BZ droplet with higher activity
or sodium bromate concentration.
% When the two BZ droplets come together, they show collective
% motion. During this collective motion, bromine ions react with
% monoolein from the oil medium and reduce the interfacial tension in
% that region. Due to a lowering in interfacial surface tension, BZ
% droplets move away after some interval of time.  In acid medium
% \ch{NaBrO_3} produces the \ch{Br^-} ions.  When \ch{NaBrO_3}
% concentration increases, then the concentration of \ch{Br^-} ions
% also increases. Which increases the rate of monoolein
% bromination. Around BZ droplet quickly reduces the interfacial
% surface tension, and the droplet moves away.

\begin{figure}[H]
    \centering
    \includegraphics[width=0.8\linewidth]{Figurs/NaBrO3+different_NaBrO3_Combined.pdf}
    \caption{\textbf{Time Interval of Droplet Close Interaction:} Plot
      (a) is the interaction time of the same BZ droplets as a
      function of sodium bromate. The interaction time of the BZ
      droplets decreased, but at a higher concentration of sodium
      bromate, the change was small. Plot (b) is the BZ droplet
      interaction time for the two droplets with different
      concentrations of sodium bromate. In one droplet, the sodium
      bromate concentration was fixed (\ch{NaBrO_3} = 0.52M), and the
      other droplet varied. The interaction time of the BZ droplets
      decreases with the sodium concentration of the other
      droplets. At the higher concentration of sodium bromate in the
      other droplet, the change in interaction time was small.}
    \label{H2SO4+NaBrO3_Combined_fit}
\end{figure}

%We were also curious about the BZ droplets interaction with various
%active behaviors. To investigate this, we examined the interaction
%between a BZ droplet with a fixed sodium bromate concentration and
%another droplet with varying concentrations of sodium bromate. We
%observed similar results to those shown in Figure
%\ref{H2SO4+NaBrO3_Combined_fit} (a), where the interaction time
%decreases as the sodium bromate concentration in the second BZ droplet
%increases. The interaction time between the BZ droplets varies
%depending on the droplet has the higher concentration of sodium
%bromate.

% When we sequentially increased the concentration \ch{NaBrO_3}, the
% interfacial surface tension around the BZ droplet decreased fast,
% and the droplet moved away. But at higher concentrations of
% \ch{NaBrO_3}, the frequency of chemical wave propagation increased
% inside the droplet. The direction of chemical wave propagation was
% random. Due to the bromination of monoolein, the BZ droplet moves
% away, but at the same time, because of the chemical wave inside the
% droplet, both droplets try to move away but come close together and
% stay together for a sufficient time. Figure
% \ref{H2SO4+NaBrO3_Combined_fit} (a) shows that in low concentration
% region of \ch{NaBrO_3} with a concentration of \ch{NaBrO_3} (\ch{T_2
% - T_1}) reduces but at higher concentration, because of chemical
% wave (\ch{T_2 - T_1}) was nearly constant. The close droplet
% interaction for the BZ droplet for the different chemical
% compositions of \ch{NaBrO_3} is shown in Figure
% \ref{H2SO4+NaBrO3_Combined_fit}(b). In figure (b), the \ch{NaBrO_3}
% concentration of one droplet was 0.53M, and for the other droplet,
% 0.53M, 0.62M, 0.72, 0.84M, and 0.96M, respectively, to know the
% effect of the gradient of \ch{NaBrO_3} concentration on the close
% interaction of BZ droplet. Plot (b) shows the same behavior as
% observed in Figure \ref{H2SO4+NaBrO3_Combined_fit}(a). In figure
% \ref{H2SO4+NaBrO3_Combined_fit}(b), it is observed that the close
% interaction of BZ droplets depends on the BZ droplet, which has a
% higher concentration of \ch{NaBrO_3} than the lower concentration BZ
% droplet.  Similarly, we studied the interaction of BZ droplets of
% different volumes. We observed two different types of
% interaction. In Figure \ref{1&5_combined}, Plot (a) and (b) show the
% path following interaction. When a big droplet follows a small
% droplet, this interaction lasts for a long time. When two BZ
% droplets show interaction from opposite directions, or a small
% droplet follows a big droplet, then the interaction time of the
% droplets is small compared to the first type, as shown in Figure
% \ref{1&5_combined}. When a small BZ droplet follows the big droplet,
% the big droplet, which has a higher surface area, reduces the large
% interfacial surface tension in the path. Due to lower interfacial
% surface tension, small droplets following big droplet interaction
% will not sustain for a long time, as observed when a big droplet
% follows a small droplet.
%\begin{figure}[H]
   % \centering
    %\includegraphics[width=1\linewidth]{Figurs/1&5_Combined12.pdf}
    %\caption{\textbf{BZ Droplet Interaction for Different Size:} The
    % plot presents the trajectory and the distance between the
    % centers of the droplets when two droplets of different sizes
    % interact. Plots a and b show interaction when a big droplet
    % follows the small droplet. This interaction of droplets lasted
    % for a longer time. Plots (c) and (d) present the trajectory and
    % distance between the center of the droplet when the droplets
    % interact during the parallel motion. This interaction was for a
    % short time compared to (a). In the plot, the interaction was
    % plotted of 1 $\mu$L and 5$\mu$L.}
    %\label{1&5_combined}
%\end{figure}

\begin{figure}[H]
    \centering
    \includegraphics[width=1\linewidth]{Diff_Size_Trajectory_Circle.pdf}
    \caption{\textbf{Different Size BZ Droplet Interaction :} Plots a,
      b, and c are the interaction trajectories of the two
      different-sized droplets when both droplets interact through
      parallel motion, opposite direction, and path following
      motion. From the figure, we can observe that the c type of
      interaction had a higher time of interaction than plots a and
      b. }
    \label{Different_size}
\end{figure}

We also investigated how droplets of different sizes interact with one
another. We observed three distinct types of interactions based on how
the droplets approach each other. The droplets can either arrive from
opposite directions, move in parallel, or have a larger droplet trail
behind a smaller one. Larger BZ droplet has the higher speed than the smaller BZ droplet. When
a small droplet approaches from the opposite direction, or when both
droplets move parallel while maintaining a minimum separation, the
interaction time is shorter than when a larger droplet follows a
smaller one. In the case where the large droplet trails the small one,
its speed is suppressed by the smaller droplet, leading to a sustain
interaction for long time. These three interaction types can be seen
in Figure \ref{Different_size}.
\begin{figure}[H]
    \centering
    \includegraphics[width=0.6\linewidth]{Close_long_interaction.pdf}
    \caption{\textbf{Interaction Time of Different-size Droplet
        Interactions:} The plots present the interactions of
      different-sized droplets with interaction time. The shaded
      region presents the fig \ref{Different_size}(a) and (b)
      interaction, when droplets arrive from opposite directions, or
      move parallel with minimum distance. The other part contained
      the interactions, which had an interaction time of 50 seconds or
      more. In this interaction region, a small droplet was always
      followed by a big droplet as observed in Figure
      \ref{Different_size}(c).}
    \label{Big-Small Droplet Interaction Time}
\end{figure}

When two droplets with different size approach from opposite
directions and move parallel to each other, their interaction time is
typically less than 50 seconds, as indicated in the shaded area of
Figure \ref{Big-Small Droplet Interaction Time}. Conversely, in other
sections of the figure, the interaction time exceeds 50 seconds when a
larger droplet trails a smaller one. When the smaller droplet follows
the larger on, the interaction time was negligible. The larger
droplet, moving at a higher speed. When smaller droplet fallows the larger
droplet, larger droplet easily moved away of interaction and
interaction was not sustained. Additionally, when two droplets of
equal size interact, the time of interaction remains consistent
regardless of their approach sides. The collective interaction of the
BZ droplets was noted within a 2.5 mm or less distance between them,
specifically for BZ droplets with a diameter of 1.75 mm.
%\begin{figure}[H]
%  \centering
%  \includegraphics[width=0.6\linewidth]{Figurs/Comparison.pdf}
%  \caption{\textbf{Number of Interactions of Both Types:} The plot
%  presents the number of interactions of long-time interaction and
%  short-time interaction out of 100\%. Here, 100\% is the 30 number
%  of interactions. The short-time interaction dominates in the number
%  of times interaction, while long-time interaction dominates in
%  interaction time.}
    %\label{Interaction Bar Plot}
%\end{figure}
%We analyzed the close droplet interaction for BZ droplet for the
% different chemical compositions of \ch{NaBrO_3}as shown in figure
% \ref{different_chemicals}. In figure (a), the \ch{NaBrO_3}
% concentration of one droplet was 0.53M, and for the other droplet,
% 0.53M, 0.62M, 0.72, 0.84M, and 0.96M, respectively. Plot (a) shows
% the same behavior as observed in figure
% \ref{H2SO4+NaBrO3_Combined_fit}. In figure
% \ref{different_chemicals}, it is observed that the close interaction
% of BZ droplets depends on the BZ droplet, which has a higher
% concentration of \ch{NaBrO_3} than the lower concentration BZ
% droplet.

%\begin{figure}[H]
%  \centering
%  \includegraphics[width=1\linewidth]{Figurs/2H2SO4+2NaBrO3_Combined.png}
%  \caption{\textbf{BZ Droplet Interaction for Different Chemical
%  Concentration
%  Containing %Droplet:} Plot (a) presents the 2 $\mu$l BZ droplet interaction time interval for the 0.53M and varying concentration of \ch{NaBrO_3}. Plot (b) shows the 2$\mu$l BZ droplet interaction for the BZ droplet, which is 0.5M and different varying concentrations of \ch{H_2SO_4}.}
    %\label{different_chemicals}
%\end{figure}

\section{Discussion}
It is well known that the chemical oscillations inside the BZ droplet
and the interfacial tension around the BZ droplet cause the
self-propelled motion of the BZ droplet. The change in interfacial
tension around the BZ droplet is influenced significantly by the
surfactant monoolein. The \ch{Br^-} ions produced in the BZ droplet
react with monoolein from the oil medium, and this bromination of
monoolein reduced the interfacial tension around the BZ droplet.

Sodium bromate is source \ch{Br^-} ions in the BZ reaction. At
higher concentrations of sodium bromate, the BZ reaction produced a
large number of \ch{Br^-} ions. As the sodium bromate concentration
increases in the BZ droplet, the rate of bromination of monoolein also
increases, and as a result, the speed of the BZ droplet increases.


In the interaction of two BZ droplets, the BZ droplets interact
until the surrounding interfacial tension is reduced
sufficiently. When the sodium bromate concentration increased in the
BZ droplets, the rate of reduction in interfacial tension around BZ
droplets increased due to bromination of surfactant monoolein from the
oil. As a result, the interaction time of the BZ droplet decreased
with the sodium bromate concentration.

\section{Conclusion}
The self–propelled motion of the BZ droplet is associated with
chemical oscillations inside the droplet and the interfacial tension
around the BZ droplet. The sodium bromate concentration inside the BZ
droplet plays a vital role in the self-propelled motion of the BZ
droplet. In the self-propelled motion of the BZ droplet, fluctuations
in the direction of motion, frequency of oscillations, and speed of
droplet increase as a function of sodium bromate. It confirmed that
the activity of the BZ droplet increases with the sodium bromate
concentration. In this experimental work, we explored the interaction
of BZ droplets as a function of sodium bromate. Two BZ droplets,
during self-propelled motion, moves in a random direction, come
together and show the collective interaction behavior. This BZ droplet
shows the collective interaction and motion, with the inter-droplet
distance of around 2.5 mm or less. The interaction time of the BZ
droplets decreases with the sodium bromate concentration. To explore
the interaction of the BZ droplet with different activities, we fixed
the sodium bromate concentration of one droplet and varied it in the
other BZ droplet; the interaction time decreases as the sodium
concentration increases in the other droplet. The interaction time of
the droplets depends on the droplet with higher sodium bromate. When
two BZ droplets of different sizes interact, three types of
interactions are observed. Two BZ droplets arrived from opposite
directions, moved in parallel, and the big droplet followed the small
droplet. The first two have the interaction time less than 50
seconds. When a big droplet follows a small droplet, the interaction
time is always higher. The interaction time of the BZ droplet depends
on the droplet with the highest activity. Big droplet has higher
speed, but when they follow a small droplet, their activity gets
suppressed, and the interaction lasts for a long time.
\par Similarly, BZ droplet activity can be controlled by the sulphuric
acid of the BZ droplet. In acidic medium, sodium bromate produces
\ch{Br^-} ions. These \ch{Br^-} ions are helpful for the
self-propelled motion and oscillations of the BZ droplet. We observed
nearly the same result with sulphuric acid as observe with sodium bromate. The interaction time of
the BZ droplet decreases with the sulphuric acid. When two droplets
with different sulphuric acid concentrations interact, the interaction
time depends more on the sulphuric acid concentration of the droplet.

\par In this work, we explored the interaction of BZ droplets when
both droplets are sufficiently close. To know what forces attract the
droplet from a long distance, it is still not clear. We hope
our findings might help researchers to simulate the interaction of an
active system by considering the internal energy of the active system.
% The self–propelled motion of the BZ droplet is associated with
% chemical composition inside the droplet, and the interfacial tension
% around the BZ droplet. Inside the droplet, the BZ reaction produces
% Br- ions, which react with monoolein, which is present in an oil
% medium. Because of the bromination of monoolein, interfacial surface
% tension reduces, and the droplet moves from a lower interfacial
% surface tension to a higher interfacial surface tension. While
% moving randomly, two droplets come close and show collective
% behavior. In this work, we explore the close interaction of BZ
% droplets.  Due to the bromination of monoolein from both droplets,
% when the interfacial tension reduces sufficiently, the droplets move
% away from each other. The concentration of the sodium bromate
% increases the concentration of Br- ions, which leads to an increased
% rate of bromination of monoolein and a decrease in close interaction
% time (\ch{T12}). At higher concentrations of \ch{NaBrO_3}, the
% frequency of chemical oscillations increases. The direction of
% propagation of the chemical wave was random. When two droplets with
% higher concentrations were interacting because of interfacial
% surface tension, droplets tried to move away, but at the same time,
% because of chemical wave orientation, they came together and showed
% collective motion, even maintaining some minimum distance. With the
% sodium bromate concentration, close interaction time decreases, but
% at higher concentrations, due to chemical waves, it becomes nearly
% constant. Similarly, in a higher acidic medium, large Br- ions are
% produced in the BZ reaction, which increases the bromination of
% monoolein. The close interaction time of the BZ reaction is nearly
% reduced with \ch{H_2SO_4}.In the ballistic motion of the BZ droplet,
% the close interaction time interval was maximum, while in the random
% motion, the close interaction was lower compared to the ballistic
% motion. In different-sized droplets, when a big droplet is a small
% droplet, then it is for a long time, while other interactions last
% for a shorter time. Based on this study, we can infer that the
% internal chemical reactions significantly influence the collective
% movement of droplets in the presence of a chemical gradient. We
% anticipate that the close interactions within the active BZ droplet
% system will serve as a valuable model for investigating the
% interactions of biomimetic soft swimmers.

\newpage
\section{Appendix}

In the Belousov-Zhabotinsky (BZ) reaction, an acidic environment is necessary to produce \(\text{Br}^-\) ions from sodium bromate, which we achieve by using sulfuric acid. An increase in the concentration of sulfuric acid leads to a higher rate of \(\text{Br}^-\) production. This results in greater availability of \ch{Br^-} ions for the chemical oscillation and self-propelled motion of BZ droplets. The speed of the droplet, the fluctuation in the direction of motion, and the frequency of oscillations inside the droplet increase with the sulphuric acid. The effect of sulphuric acid on the frequency of oscillations was small compared to that of sodium bromate.
%%%%%%%%%%%%%%%figure
\begin{figure}[H]
    \centering
    \includegraphics[width=1\linewidth]{Figurs/Combined_activity_H2SO4.pdf}
    \caption{\textbf{Activity of BZ Droplet with Sulphuric Acid:} Plots a, b, and c are the frequency, speed, and fluctuation in the direction of the BZ droplet with sulphuric acid, respectively. The frequency of oscillations inside the droplet, the speed of motion, and fluctuations in the direction of motion increase with sulphuric acid.}
    \label{Activity_H2SO4}
\end{figure}

We measure the inter-droplet distance of the interacting BZ droplet for different concentrations of sulphuric acid with time. The dotted line presents the interaction time of the BZ droplet when it is 2.5mm or less.
\begin{figure}[H]
    \centering
    \includegraphics[width=1\linewidth]{Figurs/Sulphuric_R12_combined.pdf}
    \caption{\textbf{Inter Droplet Distance with Sulphuric Acid:} Plot (a), (b), and (c) is the distance between the centers of droplets with time for the different sulphuric acid concentrations of 0.5M, 0.7 M, and 0.8M, respectively. The sodium bromate concentration was 0.48M. The dotted line presents the interaction time when the inter-droplet distance was 2.5mm or less for the 1.75mm diameter.}
    \label{H2SO4_R12}
\end{figure}
When the concentration of sulphuric acid increases in the BZ droplet, the interaction time of the BZ droplet decreases. In Figure \ref{fig:dropletDistance}(d), at a higher concentration of sodium bromate, due to chemical wave interaction, the interaction time was not varied much. In this interaction, the effect of the chemical wave was not observed notably in the interaction with sulphuric acid. The interaction decreased linearly.
\begin{figure}[H]
    \centering
    \includegraphics[width=1\linewidth]{Figurs/H2SO4+different_H2SO4_Combined.pdf}
    \caption{\textbf{Interaction Time Interval for Sulphuric Acid:}  Plot (a) is the interaction time for the concentration of \ch{H_2SO_4}. For plot (a), interacting droplets are chemically identical. The interaction time of the BZ droplets decreases with sulphuric acid. Sodium bromate concentration fixed at 0.48M. Plot (b) is the BZ droplet interaction of two BZ droplets with different sulphuric acid concentrations. In one droplet, the sulphuric acid concentration was 0.5M, and in the other droplet, it varied. The interaction time decreases with the concentration of sulphuric acid in the other droplet. Sodium bromate concentration fixed at 0.52M.}
    \label{H2SO4}
\end{figure}
In Figure \ref{H2SO4}, we fixed the concentration of sulphuric acid in one droplet. We varied the concentration of the other droplet to verify the effect of the gradient of sulphuric acid concentration on the interaction of the droplets. The interaction time of the BZ droplets decreased with the increase in the sulphuric acid concentration of the other droplet. From Figure \ ref {H2SO4}(b), we can observe that the interaction time of the BZ droplets with different sulphuric acid concentrations depends more on the droplet with the higher concentration. 
%%%%%%%%% bibliography

\bibliographystyle{apalike}
\bibliography{Reference}

\end{document}
